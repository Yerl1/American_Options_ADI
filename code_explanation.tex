\documentclass[12pt]{article}
\usepackage{amsmath, amssymb}
\usepackage{graphicx}
\usepackage{geometry}
\usepackage{hyperref}
\usepackage{cite}
\usepackage{listings}
\usepackage{xcolor} 
\usepackage{textgreek}

\geometry{a4paper, margin=1in}
\lstdefinestyle{matlabstyle}{
    language=Matlab,
    basicstyle=\ttfamily\small,
    keywordstyle=\color{blue},
    commentstyle=\color{green},
    stringstyle=\color{red},
    numbers=left,
    numberstyle=\tiny,
    frame=tb,
    captionpos=b
}
\lstset{style=matlabstyle}
\title{On the Pricing of American Options Using ADI Schemes and Their Optimal Exercise Boundaries}
\author{Your Name \\ Astana IT University}
\date{\today}

\begin{document}

\section{Code Explanation}
\label{sec:code}

This section presents a comprehensive walkthrough of our MATLAB implementation of the predictor-corrector scheme for pricing American put options, based on the numerical framework developed by Zhu (2011). Below, we clarify the objectives, methodology, and key components of the implementation.

% ======================================
\subsection{Initialization and Parameters}
\label{subsec:init}

\begin{lstlisting}[language=Matlab,caption={Parameter setup},label=code:params]
T = 1;          % Time to maturity (years)
sigma = 0.3;    % Volatility
X = 100;        % Strike price
r = 0.1;        % Risk-free rate
D0 = 0;         % Dividend yield

% Normalized parameters
gamma = 2*r/sigma^2;
D = 2*D0/sigma^2;
\end{lstlisting}
To facilitate the numerical solution of the differential system, we introduce dimensionless variables through normalization:
    \begin{align*}
        \gamma &= \frac{2r}{\sigma^2} \\
        D &= \frac{2D_0}{\sigma^2}
    \end{align*}
where $r$ represents the risk-free interest rate, $\sigma$ denotes the volatility, and $D_0$ is the original dividend yield parameter. This non-dimensionalization simplifies the system while preserving its fundamental characteristics.
% ======================================
\subsection{Grid Construction}
\label{subsec:grid}
\begin{lstlisting}[language=Matlab,caption={Spatial/temporal discretization},label=code:grid]
x_max = log(5); 
dx = x_max/M;       
tau_exp = T * sigma^2 / 2;  
dtau = tau_exp/N;   
\end{lstlisting}
\textbf{Implementation Notes:}  
\begin{itemize}
    \item Based on Willmott et al.'s estimate, the upper bound \( S_{\text{max}} \) does not need to be excessively large—typically around three to four times the strike price. To ensure this condition is met, we set \( x_{\text{max}} = \log(5) \), corresponding to an underlying asset price approximately five times the optimal exercise price.
    \item The computational domain is discretized using uniform grids, with $M+1$ nodes in the $x$-direction and $N+1$ nodes in the $s$-direction (where $M$ and $N$ represent the number of intervals in each respective direction). For clarity in notation, we define the spatial step sizes as:
    \begin{align*}
        \Delta x &= \frac{x_{\text{max}}}{M} \\
        \Delta s &= \frac{s_{\text{exp}}}{N}
    \end{align*}
    where $s_{\text{exp}} = T\sigma^2/2$ represents the normalized contract tenor relative to half the variance of the underlying asset. The numerical solution at each grid point is denoted by $P^n_m$, where the superscript $n$ indicates the temporal discretization step and the subscript $m$ identifies the spatial grid point in the log-transformed asset domain.
\end{itemize}
\subsection{Initialization of Numerical Arrays}
The implementation begins with initialization of the solution matrices:

\begin{lstlisting}[language=Matlab, caption=Array initialization, label=code:init]
P = zeros(M+1, N+1);  % Solution matrix for option prices
S_f = zeros(N+1,1);    % Free boundary vector
S_f(1) = 1;            % Initial condition for free boundary
\end{lstlisting}

where:
\begin{itemize}
    \item $P$ is the $(M+1)\times(N+1)$ solution matrix storing option prices at each:
    \begin{itemize}
        \item Spatial grid point (rows, $m = 0,...,M$)
        \item Temporal level (columns, $n = 0,...,N$)
    \end{itemize}
    \item $S_f$ contains the free boundary values at each time step
    \item $S_f(1)=1$ sets the initial condition at $n=0$
\end{itemize}
\subsection{Coefficient Computation}
The coefficients $\alpha$ and $\beta$ implement the finite difference scheme's discretization weights:

\begin{lstlisting}[language=Matlab]
alpha = 1 + (gamma/2)*dx^2;       % Curvature term
beta = 1 + dx + (D + 1)*dx^2/2;   % Combined drift and boundary effects
\end{lstlisting}

These derive from the paper's discretization where:
\begin{itemize}
\item $\alpha = 1 + \frac{\gamma}{2}\Delta x^2$ handles second-order price sensitivity (gamma)
\item $\beta$ combines:
  \begin{itemize}
  \item $1$: Base value
  \item $\Delta x$: First-order price change (drift)
  \item $\frac{D+1}{2}\Delta x^2$: Free boundary coupling ($D$) and convexity adjustment
  \end{itemize}
\end{itemize}
The coefficients link the free boundary $S_f$ to adjacent node values in the predictor-corrector scheme.
% ======================================
\subsection{Predictor-Corrector Scheme}
The implementation follows the two-phase approach:

\begin{enumerate}
    \item \textbf{Predictor}: Explicit Euler scheme for boundary estimation
    \item \textbf{Corrector}: Crank-Nicolson scheme for price calculation
\end{enumerate}

\subsection{Predictor Step: Explicit Euler Scheme}
\label{subsec:predictor}

The predictor step estimates the optimal exercise boundary \(S_f(\tau)\) using an explicit Euler discretization of the transformed PDE. The implementation computes spatial derivatives and linearizes the nonlinear term for efficient boundary estimation.

\subsubsection{Spatial Derivatives Computation}
Central finite differences approximate the required derivatives:

\begin{lstlisting}[language=Matlab,caption={Derivative calculations},label=code:derivs]
% Second derivative (d²P/dx²)
C = (P_curr(3) - 2*P_curr(2) + P_curr(1))/dx^2; 

% First derivative (dP/dx) with convection coefficient
termD = (gamma - D - 1)*(P_curr(3) - P_curr(1))/(2*dx); 
\end{lstlisting}

These implement the discrete operators:
\begin{align}
\frac{\partial^2 P}{\partial x^2} &\approx \frac{P_{m+1}^n - 2P_m^n + P_{m-1}^n}{\Delta x^2} \label{eq:diffusion}\\
\frac{\partial P}{\partial x} &\approx \frac{P_{m+1}^n - P_{m-1}^n}{2\Delta x} \label{eq:convection}
\end{align}

\subsubsection{Linear and Nonlinear Terms}
The PDE's linear reaction term and nonlinear boundary term are computed as:

\begin{lstlisting}[language=Matlab,caption={Term evaluations},label=code:terms]
E = gamma * P_curr(2);    
F = (P_curr(3) - P_curr(1))/(2*dx*S_f(n)); % Nonlinear coefficient
\end{lstlisting}

Corresponding to:
\begin{equation}
\gamma P + \underbrace{\frac{\partial P}{\partial x}\frac{1}{S_f}\frac{dS_f}{d\tau}}_{\text{Linearized as } F \cdot \Delta S_f/\Delta\tau} \label{eq:pde_terms}
\end{equation}

\subsubsection{Boundary Estimation}
The boundary prediction combines the discretized PDE:

\begin{lstlisting}[language=Matlab,caption={Boundary predictor},label=code:boundary_pred]
S_f_hat = (alpha - P_curr(2) + F*S_f(n) - (C + termD - E)*dtau)/(F + beta);
\end{lstlisting}

Derived from:
\begin{equation}
S_f^{n+1} = \frac{\alpha - P_1^n + F S_f^n - (\mathcal{L}P^n)\Delta\tau}{F + \beta} \label{eq:boundary_update}
\end{equation}
where \(\mathcal{L}P^n\) represents the discretized PDE operator from \eqref{eq:diffusion}-\eqref{eq:pde_terms}.

\subsubsection{Stability Considerations}
The explicit scheme requires:
\begin{equation}
\frac{\Delta\tau}{\Delta x^2} \leq 1 \quad \text{(CFL condition)} \label{eq:cfl}
\end{equation}
The nonlinear term treatment ensures:
\begin{itemize}
\item First-order temporal accuracy
\item Linear computational complexity
\item Conditional stability maintained by \eqref{eq:cfl}
\end{itemize}

\begin{table}[h]
\centering
\caption{Predictor step variable mapping}
\begin{tabular}{lll}
\hline
Code Variable & Mathematical Equivalent\\ \hline
\texttt{C} & \(\frac{\partial^2 P}{\partial x^2}\) \\
\texttt{termD} & \((\gamma-D-1)\frac{\partial P}{\partial x}\) \\
\texttt{E} & \(\gamma P\) \\
\texttt{F} & \(\frac{\partial P}{\partial x}\frac{1}{S_f}\)\\
\texttt{S\_f\_hat} & \(S_f^{n+1}\)\\ \hline
\end{tabular}
\end{table}
\subsection{Corrector Step Implementation}
\label{subsec:corrector}

The corrector step refines the solution using a Crank-Nicolson discretization of the linearized PDE system from. This implicit scheme provides second-order temporal accuracy while maintaining numerical stability.

\subsubsection{Matrix Formulation}
The discrete system is expressed as:

\begin{equation}
AP_{m}^{n+1} = BP_{m}^{n} + e
\label{eq:matrix_system}
\end{equation}

where:
\begin{itemize}
\item $P_{m}^{n+1} = (P_{1}^{n+1}, \ldots, P_{M-1}^{n+1})^{T}$ is the unknown vector at time $n+1$
\item $P_{m}^{n} = (P_{1}^{n}, \ldots, P_{M-1}^{n})^{T}$ contains known values at time $n$
\item $e$ incorporates boundary conditions
\end{itemize}

\subsubsection{Matrix Construction}
The tridiagonal matrices $A$ and $B$ are built using coefficients
\begin{lstlisting}[language=Matlab,caption={Matrix coefficient computation},label=code:coefficients]
% Spatial and temporal weights
a = (dtau/(2*dx^2)) - ((gamma-D-1)*dtau)/(4*dx) - (dSf)/(2*dx*Sf_avg);
b = 1 + dtau/dx^2 + (gamma*dtau)/2;
b_prime = 1 - dtau/dx^2 - (gamma*dtau)/2;
c = (dtau/(2*dx^2)) + ((gamma-D-1)*dtau)/(4*dx) + (dSf)/(2*dx*Sf_avg);
\end{lstlisting}

These coefficients correspond to:
\begin{align}
a &= \frac{\Delta\tau}{2\Delta x^2} - \frac{\gamma - D - 1}{4}\frac{\Delta\tau}{\Delta x} - \frac{1}{2\Delta x}\frac{\Delta S_f}{S_f^{avg}} \label{eq:coeff_a}\\
c &= \frac{\Delta\tau}{2\Delta x^2} + \frac{\gamma - D - 1}{4}\frac{\Delta\tau}{\Delta x} + \frac{1}{2\Delta x}\frac{\Delta S_f}{S_f^{avg}} \label{eq:coeff_c}
\end{align}

where $S_f^{avg} = (S_f^{n+1} + S_f^n)/2$ and $\Delta S_f = S_f^{n+1} - S_f^n$.
\subsubsection{System Solution}
The linear system is solved efficiently using MATLAB's sparse solver:

\begin{lstlisting}[language=Matlab,caption={Linear system solution}]
% Construct tridiagonal matrices
A = spdiags([-a*ones(M-1,1), b*ones(M-1,1), -c*ones(M-1,1)], -1:1, M-1, M-1);
B = spdiags([a*ones(M-1,1), b_prime*ones(M-1,1), c*ones(M-1,1)], -1:1, M-1, M-1);

% Solve system
P_next_inner = A \ (B*P(2:M,n) + e);
\end{lstlisting}

\subsubsection{Convergence Analysis}
The scheme achieves:
\begin{itemize}
\item \textbf{Second-order accuracy} in space ($O(\Delta x^2)$) via central differences
\item \textbf{Unconditional stability} through the Crank-Nicolson method
\item \textbf{Optimal complexity} ($O(M)$ operations) using tridiagonal solver
\end{itemize}

\begin{table}[h]
\centering
\caption{Corrector step component summary}
\begin{tabular}{lll}
\hline
Component & Mathematical Basis \\ \hline
Matrix $A$ & Implicit terms  \\
Matrix $B$ & Explicit terms \\
Vector $e$ & Boundary conditions  \\
Solver & Tridiagonal algorithm \\ \hline
\end{tabular}
\end{table}
\subsubsection{Linear System Solution}
The system is solved using the MATLAB backslash operator:

\begin{lstlisting}[language=Matlab,caption={System solution}]
P_next_inner = A \ (B*P(2:M,n) + e);  
\end{lstlisting}

\subsubsection{Coefficient Derivation}
The matrix coefficients are derived from:

\begin{align*}
a &= \frac{\Delta\tau}{2\Delta x^2} - \frac{\gamma - D - 1}{4}\frac{\Delta\tau}{\Delta x} - \frac{1}{2\Delta x}\frac{S_f^{n+1} - S_f^n}{S_f^{n+1} + S_f^n} \\
b &= 1 + \frac{\Delta\tau}{\Delta x^2} + \frac{\gamma}{2}\Delta\tau \\
b' &= 1 - \frac{\Delta\tau}{\Delta x^2} - \frac{\gamma}{2}\Delta\tau \\
c &= \frac{\Delta\tau}{2\Delta x^2} + \frac{\gamma - D - 1}{4}\frac{\Delta\tau}{\Delta x} + \frac{1}{2\Delta x}\frac{S_f^{n+1} - S_f^n}{S_f^{n+1} + S_f^n}
\end{align*}

\begin{table}[h]
\centering
\caption{Coefficient physical meanings}
\begin{tabular}{ll}
\hline
Coefficient & Physical Meaning \\ \hline
$a$ & Weighting of lower diagonal terms \\
$b$ & Main diagonal (diffusion + reaction) \\
$c$ & Weighting of upper diagonal terms \\ \hline
\end{tabular}
\end{table}

\subsection{Boundary Conditions}
Implements conditions:
\begin{lstlisting}[caption={Boundary condition implementation}]
% Value-matching condition 
P(1, n+1) = 1 - S_f(n+1);

% Gamma condition 
S_f(n+1) = (alpha - P(2,n+1)) / beta;
\end{lstlisting}
\end{document}